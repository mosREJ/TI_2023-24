\documentclass{article}
\usepackage{graphicx} % Required for inserting images

%============pcgs===================================
\usepackage{geometry}
\usepackage[utf8]{inputenc}
\usepackage{polski}
\usepackage[normalem]{ulem}
%\usepackage{mhchem}
\usepackage{url}
\usepackage{color}
\usepackage{lastpage}
\usepackage{subcaption}
\usepackage{amssymb}
\usepackage{fancyhdr}
\usepackage{multirow}
\usepackage{todonotes}
\usepackage{booktabs}
\usepackage{float}
\usepackage{siunitx}
\usepackage[export]{adjustbox}
%\usepackage{tabularx}f
\usepackage{makecell}
\usepackage{amsmath,amssymb}
\usepackage{listings}
\usepackage{graphicx}
\graphicspath{ {./photo/} }
\usepackage[justification=centering]{caption}
\usepackage{lscape}
\usepackage{sectsty}
\usepackage[T1]{fontenc}
\usepackage{mathptmx}
\usepackage{parskip}
\usepackage{listings}
\usepackage{indentfirst}
\usepackage{etoolbox}
\usepackage[shortlabels]{enumitem}
\usepackage[toc]{glossaries}
\usepackage{amsmath}
\usepackage{tocloft}
\usepackage[final]{pdfpages}
\usepackage[ampersand]{easylist}
\usepackage{float}
\usepackage{amsmath}
\usepackage[final]{pdfpages}
\usepackage[ampersand]{easylist}



\title{TI}
\author{249580@student.pwr.edu.pl }
\date{October 2023}

\begin{document}

\maketitle

\newpage 

\section{Zadanie nr 1 - wzory}

\begin{equation}
    (\sqrt{3} - \sqrt{2})^2 = (\sqrt{3})^2 - 2\sqrt{3}\sqrt{2} + (\sqrt{2})^2 = 5 - 2\sqrt{6}, 
\end{equation}

\begin{equation}
    \frac{\sqrt[3]{40}}{\sqrt[3]{5}} = \sqrt[3]{\frac{40}{5}} = \sqrt[3]{8} = 2, 
\end{equation}


\begin{equation}
    x_{n+1} = x_{n} + y_{n},  y_{n+1} = 4x_{n} - 2y_{n} +1,
\end{equation}

\begin{equation}
    \left\{ \begin{array}{}
        x_{n+1} = x_{n} + y_{n}\\ y_{n+1} = 4x_{n} - 2y_{n} +1
\end{array}\left
\end{equation}

\begin{equation}

    H_{c} = \frac{1}{2n} \sum_{l=0}^{n} (-1)^{l}(n-l)^{p-2}
    
\end{equation} 

\begin{equation}
     \prod_{n=10}^{10} (n+3) 
\end{equation}


\begin{equation}
    \int_{2}^{3} x \,dx
\end{equation}

\begin{equation}
    
\end{equation}

 
\section{Zadanie nr 2 - tablica, macierz}




$\begin{bmatrix}
2+\lambda & 1 & 3 \\
3 & -6+\lambda & 5\\
9 & -2 & -3+\lambda 
\end{bmatrix}  




\begin{table}[ht]
\begin{tabular}{|ll||l||l||l||}
\hline
\multicolumn{1}{||l||}{123}     & 23      & 34     \\ \hline
\multicolumn{2}{||l||}{Połączone komórki} & 2      \\ \hline
\multicolumn{1}{||l||}{1}       & 2247    & 3555  \\ \hline
\end{tabular}
\end{table}



% Please add the following required packages to your document preamble:
% \usepackage{booktabs}
\begin{table}[ht]
\begin{tabular}{@{} l l l l @{}}
\toprule
\toprule
Case & Method \#1 & Method\#2 & Method\#3 \\ 
1    & 50         & 837       & 970       \\ 
2    & 47         & 877       & 230       \\ 
3    & 31         & 25        & 415       \\ 
4    & 35         & 144       & 2356      \\ 
5    & 45         & 300       & 556       \\ \bottomrule
\end{tabular}
\end{table}

\section{Zadanie nr 3 - dodanie zdjęcia}

    \begin{figure}[h]
        \centering
        \includegraphics{photo/logotypy-pwr.png}
        \caption{Logotypy uczelni}
        \label{fig:photo_LogotypyUczelni}
    \end{figure}



\end{document}

